\chapter{Contribuciones}
\section{Colin Ulrich Cop}
Antes de empezar el proyecto, tuve mis primeros pasos desarrollando aplicaciones de realidad aumentada con marcadores usando la librería Vuforia. Más tarde, adquirí un teléfono que soportaba ARCore por lo que empecé a probar la realidad aumentada sin marcadores.\\

Las primeras aplicaciones desarrolladas consistían en instanciar diferentes objetos y hacer que interactuasen entre ellos. Empecé a investigar sobre las librerías disponibles y las iba probando, sobretodo 8thWall,que aún no tenía el SDK abierto y tuve que enviar una solicitud personal, con esta librería, también pude probar la realidad aumentada en la web. Durante el curso, en la asignatura \textit{Videojuegos en dispositivos Móviles}, impartida por Pedro Pablo Gómez Martin, tuvimos que programar el juego Bombardero Amstrad-CPC, una vez terminada la práctica, me pareció buena idea desarrollar dicho juego en realidad aumentada. Una vez implementadas las mecánicas básicas, el desarrollo se quedó pausado. También hice el prototipo de JengAR visto anteriormente, que no lo terminé ya que debido a su baja complejidad, no íbamos a implementar la versión multijugador. Otro de los prototipos que realicé fue un juego plataformas en el que se controlaba con el mando y se apuntaba con la cámara, estaba pensado para ser jugado con un mando bluetooth y un soporte del móvil para dicho mando. \\

Cuando descubrimos la existencia de los \textit{cloud anchors}, aún no disponíamos de dos dispositivos en los que poder probar esa tecnología, pero decidimos que una de las pruebas de concepto iba a ser un juego multijugador \textit{online} usando los \textit{cloud anchors}. Me empecé a informar respecto a cómo hacer un juego multijugador en Unity, por lo que estuve aprendiendo la \textit{Multiplayer High Level API}, y por otro lado también usé el plugin Photon\cite{pun}, porque vi que se usa mucho y es más completo que la API de Unity3D. Me encargué de crear las aplicaciones usando todas las librerías, exceptuando ARKit y Vuforia, para el posterior análisis de cada una. Dicho análisis fue realizado junto a Patricia, los dos nos reunimos para probar todas las aplicaciones en el mismo entorno y mismas condiciones. Junto a Patricia encontré las gafas Aryzon\cite{Aryzon}, y se nos ocurrió crear una aplicación en la que pudieras observar y manipular objetos 3D.\\

Cuando la facultad nos proporcionó un teléfono que soportase ARCore, pude empezar la prueba de concepto BombARdero+, que se pudo adaptar bien a los cloud anchors gracias a que ya tenía un proyecto con los prefabs y scripts hechos anteriormente. Patricia me ayudó a adaptar dicho proyecto al multijugador y a implementar las funcionalidades requeridas para que funcionase correctamente \textit{online}.\\

Con respecto a la memoria, antes de empezar a escribir, me informé en varios libros y webs sobre la historia de la RA, ya que me encargué de escribir la definición e historia. Por otro lado, tuve que estudiar varias tecnologías implicadas en la RA sin marcadores,como el SLAM, detección de planos, reconocimiento del ambiente, oclusión, estimación de luces, e hice un estudio completo sobre las funcionalidades y licencias de cada librería que hemos visto . Patricia y yo hemos realizado el capítulo 3 entero, donde tuvimos que someter cada una de las aplicaciones a un test y evaluar su rendimiento en diferentes condiciones de luz. Finalmente, junto a Patricia, traduje la introducción y las conclusiones al inglés.
\newpage
\section{Patricia Cabrero Villar}
Mi primer contacto con la realidad aumentada fue cuando realicé mi trabajo fin de grado cuando cursé el Grado de Diseño Gráfico y Multimedia, donde desarrollé una aplicación de realidad aumentada con marcadores específica para niños con trastorno del espectro autista.
Después de esta primera aproximación al mundo de la realidad aumentada me pareció interesante continuar formándome y aprendiendo en lo que refiere a este campo, con lo que decidí proponer como tema la realidad aumentada sin marcadores.\\

Una vez establecido el tema investigué en busca de tecnologías disponibles y bibliografía actualizada para poder tener una base sólida sobre la que apoyarme para el desarrollo del proyecto. De esta manera surgieron títulos como \textit{Handbook of Augmented Reality} de Carmigniani y Furth que me ayudó a comprender con mayor profundidad la tecnología a tratar y como punto de  referencia del paradigma actual establecí el libro \textit{Augmented reality games II, The gamification of education,medicine and art} de Geroimenko.\\

Realizada la investigación fui la encargada de establecer la estructura y plan de trabajo que seguiríamos a lo largo del proyecto para poder llevar a cabo los objetivos propuestos conjuntamente. En un primer paso plasmamos entre los tres los conocimientos adquiridos en la fase de investigación mostrando los antecedentes, historia y librerías de realidad aumentada en el primer y segundo capítulo.\\

Más tarde definí las pruebas y características que observaríamos en los \textit{test} de las diferentes librerías junto con mi compañero Colin. El desarrollo de las aplicaciones se dividió de manera que yo realicé las apps de Vuforia y ARKit ya que era necesario un dispositivo iOS para su correcto funcionamiento. Una vez tuvimos desarrolladas todas las pruebas de las librerías se probaron todas bajo las mismas condiciones de luz y tecnología, esta evaluación fue llevada a cabo por Colin y por mí. Estableciendo conjuntamente las puntuaciones y conclusiones de cada librería. Gracias a estas pruebas Colin y yo pudimos establecer una evaluación global de las librerías actuales y con estas conclusiones decidir las pruebas de concepto que desarrollaríamos en una segunda fase.\\

Por último, en esta segunda fase se decidió desarrollar una aplicación multijugador(BombARdero+), una inmersiva(Aryzon) y una descriptiva(AmueblAR). El desarrollo de estas aplicaciones se decidió dividir entre los integrantes ya que permitía un desarrollo más ágil y debido a la necesidad de dos teléfonos en el caso del multijugador. Debido a esto realicé junto con mi compañero Colin la aplicación multijugador BombARdero+ y el visualizador de objetos en 3D para las gafas Aryzon. En el caso de la aplicación multijugador focalicé mis esfuerzos en comprender el funcionamiento de la API de multijugador de Unity para poder utilizarla en consonancia con los \textit{cloud anchors}. Mi papel en el desarrollo de la visualización de objetos con las gafas Aryzon fue el de investigar acerca de la posibilidad de implementar la visión estereoscópica junto con ARCore.


\newpage
\section{David González Jiménez}
Dado que al inicio del trabajo la tecnología de mi móvil era la más anticuada y la única que no soportaba el uso de librerías que permitiesen trabajar con realidad aumentada sin marcadores, mis primeros pasos consistieron en documentarme y utilizar librerías preparadas para funcionar con marcadores, como Vuforia y ARToolKit.\\

En una primera incursión con ARToolkit pude encontrar un paquete que lo integraba en Unity y una documentación bastante precaria sobre cómo utilizar este \textit{plugin}. Tras unas cuantas búsquedas conseguí hacer una aplicación muy básica que funcionase en mi dispositivo a modo de aproximación, que está explicada en la sección correspondiente a las pruebas con ARToolKit.\\

Ahora que ya había tocado los orígenes de la realidad aumentada podía ponerme manos a la obra con la librería
Vuforia, también integrada en Unity. Por este entonces seguía sin tener acceso a tecnología que permitiese ejecutar aplicaciones de realidad aumentada sin marcadores, de manera que los prototipos que realicé en con esta librería también utilizaban marcadores. Sin embargo, los resultados fueron mucho más satisfactorios y vistosos porque Vuforia pone a la disposición del desarrollador gran cantidad de herramientas que hacen el trabajo más sencillo y dan mejores resultados. Los prototipos que realicé en este caso fueron dos aplicaciones para el visionado de vídeos virtuales sobre imágenes reales que se exponen en detalle en los apartados correspondientes a las pruebas con Vuforia con marcadores. Pude hacer con éxito un cartel animado basado en la saga de Harry Potter y una página de cómic de DragonBall en la que se reproduce un pequeño vídeo en cada viñeta.\\

En última instancia, y con el fin de hacer una aplicación algo más compleja con marcadores hice la implementación de un juego de cartas por turnos en la que las criaturas que están dibujadas en los naipes aparecen como modelos 3D al verlas a través de un móvil. El juego consta de turnos y sería multijugador de haber insistido en dejarlo completamente cerrado, sin embargo, al ser un prototipo de aproximación el rival es controlado por una inteligencia artificial. Todo este desarrollo puede leerse en la sección correspondiente en el capítulo del juego implementado en realidad aumentada con marcadores.\\

Finalmente, la facultad me proporcionó un móvil que permitía por fin el despliegue de aplicaciones que usan realidad aumentada sin marcadores, así que mis compañeros y yo nos repartimos diferentes aplicaciones para realizar en ellos. Yo escogí basándome en el concepto de Ikea Place una aplicación que sirviese como manual de instrucciones para cualquier usuario que quisiera montar un mueble en su casa. La aplicación de momento consta de un sofá cuyo montaje se divide en 9 pasos diferenciados que el usuario puede pasar hacia delante y detrás, así como rotarlo o moverse alrededor de él a voluntad para facilitar la comprensión del ensamblaje. El proceso de desarrollo y la idea desarrollada se pueden consultar en la sección que habla sobre la aplicación AmueblAR.\\

Por último añadir que además del desarrollo de las aplicaciones anteriores, me he documentado sobre diferentes aspectos del funcionamiento de la realidad aumentada, como son los \textit{cloud anchors} o la detección de caras para escribir dichas secciones en la memoria, así como las diferentes aplicaciones y avances que se han producido recientemente en campos como la educación, el arte, la medicina, la publicidad o el turismo en relación a las tecnologías de la realidad aumentada, además de los diferentes métodos de \textit{tracking} existentes hasta el momento.\\

\noindent