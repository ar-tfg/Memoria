\chapter{Contribuciones}
\section{Colin Ulrich Cop}
Antes de empezar el proyecto, tuve mis primeros pasos desarrollando aplicaciones de realidad aumentada con marcadores usando la librería Vuforia. Más tarde, adquirí un teléfono que soportaba ARCore por lo que empecé a probar la realidad aumentada sin marcadores. Las primeras aplicaciones desarrolladas consistían en instanciar diferentes objetos y hacer que interactuasen entre ellos.
\newpage
\section{Patricia Cabrero Villar}

\newpage
\section{David González Jiménez}
Dado que al inicio del trabajo la tecnología de mi móvil era la más anticuada y la única que no soportaba el uso de librerías que permitiesen trabajar con realidad aumentada sin marcadores, mis primeros pasos consistieron en documentarme y utilizar librerías preparadas para funcionar con marcadores, como Vuforia y ARToolKit.\\

En una primera incursión con ARToolkit pude encontrar un paquete que lo integraba en Unity y una documentación bastante precaria sobre cómo utilizar este \textit{plugin}. Tras unas cuantas búsquedas conseguí hacer una aplicación muy básica que funcionase en mi dispositivo a modo de aproximación, que está explicada en la sección correspondiente a las pruebas con ARToolKit.

Ahora que ya había tocado los orígenes de la realidad aumentada podía ponerme manos a la obra con la librería
Vuforia, también integrada en Unity. Por este entonces seguía sin tener acceso a tecnología que permitiese ejecutar aplicaciones de realidad aumentada sin marcadores, de manera que los prototipos que realicé en con esta librería también utilizaban marcadores. Sin embargo, ésto

\noindent