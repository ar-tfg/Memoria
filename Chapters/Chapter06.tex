\setcounter{chapter}{5}
\chapter{Conclusions and future work}

Thanks to this study we are able to define an accurate evaluation of the state, capabilities and limitations of augmented reality libraries and technology, achieving our main goal.\\

Markerless augmented reality is a new technology and it is accesible to a great number of people thanks to smartphones. In the last years, augmented reality libraries have been developed that ease the AR application development, these libraries are being updated introducing new features that improve the experiencie.
 After the fulfillment of the project, we can establish several conclusions about the different
libraries mentioned in this paper, based on the research carried out during these months and
the results obtained with the tests.\\

Thanks to the developed tests, we can establish that Google and Apple lead in terms of quality. ARCore and ARKit are one of the most viable options for an application that demands stability and precision. 

It is also noteworthy that Unity3D development team is making a great effort to implement the ARFoundation API, which allows us to develop augmented reality applications for Android and iOS with the same source code.

On the other hand, we have low cost libraries that are more unstable, but are capable of working on almost all devices\cite{wikitudeInstant} and in a large percentage of environments and conditions. This makes the use of libraries, such as Vuforia, EasyAR, or Maxst, an option to use for applications that do not require such demanding stability. \\

After testing all the libraries seen above, it is clear that the stability is practically perfect in most cases. We also see a great advance in the estimation of lights, remarkable in cases such as ARCore and ARKit, providing greater realism to the immersive experience. 

We can conclude that the developers of both leading companies (Apple and Google) are disrupting the market with technologies such as occlusion and cloud anchors. Augmented reality is a very new technology, so it evolves very quickly, approximately every month and a half ARCore receives a new update; without going any further while we were doing this work, we have been able to observe many important changes in the library. \\

Thanks to an extensive documentation and tutorials created by the community, learning to develop markerless AR applications is accessible to the developer. This accessibility has allowed us to develop three applications for three different sectors and it has solved some problems and doubts that came up during their development.

\section{Future work}

As future steps, the analysis can be extended to the functionalities of augmented reality smartglasses. These devices currently dominate the market so  it is interesting to submit them to the same tests and draw conclusions from each one.

On the other hand, we have done all the tests indoor, so it is important to test the libraries in an open space and study its behaviour.

It is also important to analyze the user's experience in augmented reality. It is one of the main problems that we have identified, since most users who try an augmented reality application for the first time are often disoriented. This situation prevents them from enjoying the experience.

To complement our work, it is interesting to study other functionalities that support some libraries as facial recognition, occlusion or cloud anchors. This kind of research would extend our thesis widening the scope of the  evaluation document for augmented reality libraries.
\noindent