\chapter*{Conclusions}
\addcontentsline{toc}{chapter}{Conclusions}

After the fulfillment of the project, we can establish several conclusions about the different libraries exposed in this work, based on the research carried out during these months and the results obtained with the tests.

Thanks to this study we are able to define an accurate evaluation of the state, capabilities and limitations of augmented reality libraries and technology, achieving our main goal.

Regarding the current state of the art the project has brought new conclusions.
Thanks to the developed tests, we can establish that Google and Apple lead in terms of quality. ARCore and ARKit are one of the most viable options for an application that demands stability and precision. 

It is also noteworthy that Unity3D development team is making a great effort to implement the ARFoundation API, which allows us to develop augmented reality applications for Android and iOS with the same source code.

On the other hand, we have low cost libraries that are more unstable, but are capable of working on almost all devices\cite{wikitudeInstant} and in a large percentage of environments and conditions. This makes the use of libraries, such as Vuforia, EasyAR, or Maxst, an option to use for applications that do not require such demanding stability. \\

After testing all the libraries seen above, it is clear that the stability is practically perfect in most cases. We also see a great advance in the estimation of lights, remarkable in cases such as ARCore and ARKit, providing greater realism to the immersive experience. 

We can conclude that the developers of both leading companies (Apple and Google) are disrupting the market with technologies such as occlusion and cloud anchors. Augmented reality is a very new technology, so it advances very quickly, approximately every month and a half ARCore receives a new update; without going any further while we were doing this work, we have been able to observe many important changes in the library. \\

Thanks to an extensive documentation and tutorials created by the community, learning to develop markerless AR applications is accessible to the developer. This accessibility has allowed us to develop three applications for three different sectors.

\section*{Future work}
\addcontentsline{toc}{section}{Future work}
In our future work, we establish keeping updated on the evolution of the studied libraries for being able to develop apps with more quality. Also, we would like to take up some of the ideas we dimissed and keep working on the proof of concept we did, because we think they have a lot of utility and the videogame uses some new mechanics never seen before. We tested all libraries in a indoor enviroment, we would like to test them outdoor and see their behaviour.
\noindent