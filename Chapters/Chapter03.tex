
\chapter{Comparación y análisis de las librerías}

El análisis de las librerías se estructurará en los puntos que se describen a continuación:
\begin{itemize}
\item Calidad de la documentación y primeros pasos: en este punto evaluaremos la dificultad para realizar una aplicación básica con cada librería, desde el momento en el que se descarga el SDK, hasta que se construye la APK. 
\item Evaluación de las capacidades de la librería: en este apartado se tendrán en cuenta las funcionalidades, las tecnologías que soporta y el nivel personalización dentro de la App, es decir, hasta que nivel podemos usar la API que nos proporcionan.
\item Conclusiones: gracias al estudio realizado estableceremos unas conclusiones sobre el uso de cada librería y decidiremos si nos facilita el desarrollo de alguna prueba de concepto.
\end{itemize}


Para realizar la evaluación de las capacidades y los límites de cada librería se realizará un test que consistirá en:
\begin{itemize}
\item Instanciar un objeto.
\item Movernos alrededor de dicho objeto, para comprobar la estabilidad del punto de anclaje.
\item Realizar movimientos bruscos y veloces para ver si pierde la referencia en algún momento.
\item Hacer que pierda la referencia y comprobar el tiempo en el que vuelve a aparecer el objeto.
\item Alejarnos del objeto y ver hasta que distancia sigue funcionando.
\item Comprobar cómo se comportan las librerías con diferentes intensidades de luz.
\end{itemize}

\section{Wikitude}

\subsection{Calidad de la documentación y primeros pasos}
\subsection{Evaluación de las capacidades de la librería}
\textbf{Condiciones de luz mínimas:}\\
Esta prueba está recogida en el repositorio:\\
La sala está únicamente iluminada por un haz de luz perteneciente a una habitación situada al otro lado del pasillo.\\

Consigue instanciar el objeto dentro de un plano y posee buena iluminación. El plano falla cuando empezamos a movernos haciendo un giro de 45º no esperado, si estamos encima del modelo se pierde, si damos la vuelta completa sigue perdido. Necesitamos volver a referenciarlo por que no se recupera. La calidad del modelo y su textura es óptima para las condiciones de luz que posee. Se pierde la referencia fácilmente ante los giros. Cuando se realiza un movimiento de cámara en el que el punto de anclaje sale del \textit{frustrum}\footnote{Región cerrada del espacio que delimita los objetos que aparecen representados en la pantalla.}  y más tarde se vuelve a enfocar a él, tarda un segundo en volver a posicionar el plano y su objeto. En esta primera prueba no hemos conseguido que pierda la referencia por distancia, se repetirá con un escenario más amplio.

\begin{table}[H]
\resizebox{\textwidth}{!} {
    \centering
    \begin{tabular}{|c|c|c|c|}
    \hline
         & Luz Intensa & Luz Ambiente & Luz Mínimas \\
         \hline
        Distancia máxima de captura (m) & & &\\
        \hline
        Estabilidad del punto de anclaje & & &\\
        \hline
        Resistencia a movimientos & & & \\
        \hline
        Estimación de iluminación & & & \\
        \hline
        Tiempo de recuperación del ancla & & & \\
      \hline
    \end{tabular}
  }
    \caption{Análisis Wikitude}
    \label{tab:TWikitude}
\end{table}

\subsection{Conclusiones}

\section{ARKit}
\subsection{Calidad de la documentación y primeros pasos}

\subsection{Evaluación de las capacidades de la librería}

\subsection{Conclusiones}

\section{ARCore}
\subsection{Calidad de la documentación y primeros pasos}

\subsection{Evaluación de las capacidades de la librería}
\subsubsection{Condiciones de luz mínimas:}\\
\textbf{Esta prueba está recogida en el repositorio:}\\
La sala está únicamente iluminada por un haz de luz perteneciente a una habitación situada al otro lado del pasillo y una pequeña iluminación frontal.\\
Necesita más luz(2) para reconocer el plano. La estimación de la iluminación es excelente como podemos ver a lo largo de la prueba. Al realizar un movimiento alrededor del modelo no se pierde ni se desestabiliza en ningún momento. Al posicionar el teléfono sobre el modelo sigue estable. Las texturas del modelo se ven de manera nítida y realista. En la prueba de resistencia a movimientos bruscos no desaparece nunca ni vibra la imagen dando unos resultados óptimos. Cuando se realiza un movimiento en el que el modelo desaparece del campo de visión este no llega a desaparecer nunca del entorno virtual por lo que al volver a enfocar al punto de anclaje la transición es limpia.


\begin{table}[H]
\resizebox{\textwidth}{!} {
    \centering
    \begin{tabular}{|c|c|c|c|}
    \hline
         & Luz Intensa & Luz Ambiente & Luz Mínimas \\
         \hline
        Distancia máxima de captura (m) & & &\\
        \hline
        Estabilidad del punto de anclaje & & &\\
        \hline
        Resistencia a movimientos & & & \\
        \hline
        Estimación de iluminación & & & \\
        \hline
        Tiempo de recuperación del ancla & & & \\
      \hline
    \end{tabular}
  }
    \caption{Análisis ARCore}
    \label{tab:TARCore}
\end{table}
\subsection{Conclusiones}

\section{Vuforia}
\subsection{Calidad de la documentación y primeros pasos}

\subsection{Evaluación de las capacidades de la librería}

\subsection{Conclusiones}


\section{Kudan}
\subsection{Calidad de la documentación y primeros pasos}

\subsection{Evaluación de las capacidades de la librería}
\subsubsection{Condiciones de luz mínimas:}\\
\textbf{Esta prueba está recogida en el repositorio:}\\
La sala está únicamente iluminada por un haz de luz perteneciente a una habitación situada al otro lado del pasillo.\\

El nivel de luz de la sala no supone ningún problema para posicionar el modelo. La calidad de las texturas del objeto son malas, además no existe ninguna estimación de iluminación sobre el modelo. El anclaje alrededor del objeto con movimientos suaves es aceptable pero en el momento que nos acercamos se pierde y hay que volver a referenciarlo. La estabilidad del objeto cuando nos movemos en sus proximidades es muy mala, cambiando de tamaño sin sentido aparente. La distancia máxima de captura es de siete metros aproximadamente. La capacidad de soportar movimientos bruscos es mala, pierde totalmente la posición del ancla con resultados incorrectos e incluso a veces pierde la referencia del todo. Al sacarlo del campo de visión no lo posiciona en el mismo punto donde estaba, llegando a perder en ocasiones el punto de referencia.

\begin{table}[H]
\resizebox{\textwidth}{!} {
    \centering
    \begin{tabular}{|c|c|c|c|}
    \hline
         & Luz Intensa & Luz Ambiente & Luz Mínimas \\
         \hline
        Distancia máxima de captura (m) & & &\\
        \hline
        Estabilidad del punto de anclaje & & &\\
        \hline
        Resistencia a movimientos & & & \\
        \hline
        Estimación de iluminación & & & \\
        \hline
        Tiempo de recuperación del ancla & & & \\
      \hline
    \end{tabular}
  }
    \caption{Análisis Kudan}
    \label{tab:TKudan}
\end{table}
\subsection{Conclusiones}


\section{MaxST}
\subsection{Calidad de la documentación y primeros pasos}

\subsection{Evaluación de las capacidades de la librería}
\subsubsection{Condiciones de luz mínimas:}\\
\textbf{Esta prueba está recogida en el repositorio:}\\
La sala está únicamente iluminada por un haz de luz perteneciente a una habitación situada al otro lado del pasillo.\\

El nivel de luz de la sala no supone ningún problema para posicionar el modelo. La calidad de las texturas del objeto son notables, no existe ninguna estimación de iluminación sobre el modelo. El anclaje alrededor del objeto con movimientos suaves es aceptable pero en el momento que nos acercamos se pierde y no se llega a recuperar el punto teniendo que volver a referenciarlo.  No somos capaces de dar la vuelta al modelo completo sin perderlo. No conseguimos perderlo con la distancia. La capacidad de soportar movimientos bruscos es mejorable ya que no desaparece el modelo, pero si pierde su referencia en el espacio moviéndolo a una posición diferente. Al sacarlo del campo de visión no lo posiciona de nuevo en la mayoría de las ocasiones. Sin embargo, cuando consigue mantenerlo, en la mayoría de ocasiones se desplaza del punto correcto y muy rara vez muestra la opción correcta.

\begin{table}[H]
\resizebox{\textwidth}{!} {
    \centering
    \begin{tabular}{|c|c|c|c|}
    \hline
         & Luz Intensa & Luz Ambiente & Luz Mínimas \\
         \hline
        Distancia máxima de captura (m) & & &\\
        \hline
        Estabilidad del punto de anclaje & & &\\
        \hline
        Resistencia a movimientos & & & \\
        \hline
        Estimación de iluminación & & & \\
        \hline
        Tiempo de recuperación del ancla & & & \\
      \hline
    \end{tabular}
  }
    \caption{Análisis Maxst}
    \label{tab:TMaxst}
\end{table}
\subsection{Conclusiones}


\section{8th Wall XR}
\subsection{Calidad de la documentación y primeros pasos}

\subsection{Evaluación de las capacidades de la librería}

\subsection{Conclusiones}


\section{Easy AR}
\subsection{Calidad de la documentación y primeros pasos}

\subsection{Evaluación de las capacidades de la librería}

\subsection{Conclusiones}


\section{ARFoundation}
\subsection{Calidad de la documentación y primeros pasos}

\subsection{Evaluación de las capacidades de la librería}

\subsection{Conclusiones}


\section{ARToolKit}
Información de la librería, desarrollo de HolaMundo y conclusiones
En nuestros primeros pasos en el mundo de la realidad aumentada exploramos algunas librerías como ARToolKit, con el fin de familiarizarnos con el desarrollo de este tipo de aplicaciones.\\
ARToolKit es una de las librerías de desarrollo pioneras en el ámbito que investigamos, disponible desde el año 2004 para descargar de manera gratuita y que cuenta con más de 160.000 descargas desde entonces. Se distribuyó para diversas plataformas como SGI IRIX (que dejó de utilizarse en 2006), Linux, MacOS y Windows y fue desarrollada originalmente por el Dr. Hirokazu Kato para posteriormente pasar a manos del Human Interface Technology Laboratory en la Universidad de Washington, la de Nueva Zelanda y ARToolworks.Inc en Seattle.\\

Muchas librerías posteriores se han basado en el código de ésta para ampliar sus funcionalidades, dando lugar a algunas como ARTag (que promete mayor fiabilidad a la hora de procesar imágenes por su mejor manejo de la luz), FLARToolKit (consistente en un port en ActionScript 3), ARDesktop (que facilita la creación de interfaces) o Studierstube Tracker (que mejora sus características, pero deja de ser de código abierto).\\
Además de todas las derivaciones de ARToolKit, también podemos encontrar software no orientado a programadores como ATOMIC Authoring Tool, que permitía a cualquier usuario el desarrollo de una aplicación de realidad aumentada de manera sencilla y con una interfaz intuitiva. Esta herramienta acabó cayendo en desuso a principios de la década de 2010 debido a que ya existían librerías mejores que ARToolKit y mejores alternativas en lo que a SDK se refiere.\\

Al ser ARToolKit una de las primeras herramientas para el desarrollo de realidad aumentada, no contemplaba un uso de esta sin marcadores. Una de las mayores dificultades a las que se enfrentó fue el seguimiento del “ojo” del usuario, es decir, el foco de la cámara del dispositivo. Para saber desde qué perspectiva debía dibujar los elementos virtuales la aplicación necesitaba saber a dónde está mirando el usuario en el mundo real. La librería solventa este problema utilizando algoritmos de visión que calculan la localización y orientación de la cámara basándose en marcadores físicos en tiempo real.
Los marcadores que es capaz de identificar consisten en la mayoría de los casos en un cuadrado negro bien contrastado con un fondo e interior blancos. Además, cada marcador, para diferenciarse del resto incluye pequeñas variaciones como otras figuras geométricas dentro del cuadrado.\\

Para nuestros experimentos con ARToolKit, en lugar de utilizar la librería original, utilizamos un port de la misma para ser utilizada en Unity, que puede encontrarse actualmente en https://github.com/artoolkit/arunity5. Esta extensión nos permite el acceso a componentes como ARController y ARMarker dentro del editor.\\

Para el desarrollo de este “HolaMundo” con ARToolKit en Unity hemos seguido los siguientes pasos: creamos un gameObject  que servirá como “raíz” de la escena y otro que actuará como mánager del sistema de realidad aumentada. Al mánager le incluimos el componente ARController, que está encargado de las opciones de video y del seguimiento de los marcadores. Dentro de éste modificamos la Layer a la que debe prestar atención. \\

Por otra parte, el objeto raíz de la escena incluye la luz direccional y la cámara, y además le añadimos el script AROrigin, que permite situar espacialmente la escena. La cámara, además de su script de cámara recibe un ARCamera para poder detectar los marcadores.\\

Ahora creamos un objeto que llevará la información del marcador y le añadimos el componente ARMarker, que lleva el tag del marcador que hace las veces de identificador único. Este componente tiene dos tipos de patrones para identificar por defecto: hiro y kanji. En este caso utilizaremos el patrón “hiro”, que es el que consiste en un cuadrado negro simple.\\

Añadimos a la raíz de la escena un objeto que será contenedor del objeto 3D que queremos que aparezca cuando enfocamos al marcador y que lleva el script ARTrackedObject y dentro del campo “Marker Tag” introducimos el identificador del marcador asociado al objeto.\\

Conclusiones: si bien este sistema fue útil en su día para sentar las bases del desarrollo de programas en realidad aumentada, hoy en día no tiene mucho sentido su uso. No se encuentra casi documentación actualizada para su uso y la página web que le daba soporte (artoolkit.org) ha desaparecido. Además, sus funcionalidades son muy limitadas y su rendimiento es muy inferior al que presentan otras alternativas más actuales como Vuforia, que permite también el uso de marcadores. Sí que es cierto que una aplicación generada con esta librería y que realiza la misma funcionalidad que otra, pero generada usando Vuforia pesa ligeramente menos. Sin embargo, no consideramos que sobre todo el conjunto este dato sea lo suficientemente significativo como para plantearse su uso.\\
\section{Evaluación}
\subsection{Tabla de funcionalidades}
A continuación, hay una tabla en el que se puede comparar rápidamente las funcionalidades que tiene cada una de las librerías que vamos a analizar. Más adelante, en la parte de comparación y análisis de las librerías, evaluaremos la calidad y eficiencia de estas funcionalidades para cada librería.

\begin{table}[ht]
\resizebox{\textwidth}{!} {
    \centering
    \begin{tabular}{m{2cm}|m{2.8cm}| m{2.8cm}|m{2cm}|c|m{2.8cm}|m{2cm}|m{2cm}}
        SDK & Reconocimiento 2D & Reconocimiento 3D & Detección de planos & SLAM & Reconocimiento de rostro & Estimación de luces & Otras \\
\hline
\textbf{Wikitude} & \checkmark & \checkmark & \checkmark & \checkmark & - & \checkmark & Geo AR \\
\hline
\textbf{ARKit} & \checkmark & \checkmark & \checkmark & \checkmark & \checkmark & \checkmark & Oclusión, Cloud Anchor \\
\hline
\textbf{ARCore} & \checkmark & \checkmark & \checkmark & \checkmark & \checkmark & \checkmark & Cloud Anchor \\
\hline
\textbf{Vuforia} & \checkmark & \checkmark & \checkmark & \checkmark & - & \checkmark &  \\
\hline
\textbf{Kudan} & \checkmark & - & \checkmark & \checkmark & - & \checkmark &  \\
\hline
\textbf{MaxST} & \checkmark & \checkmark & \checkmark & \checkmark & – & \checkmark & \\
\hline
\textbf{8th Wall XR} & \checkmark & – & \checkmark & \checkmark & – & \checkmark &  \\
\hline
\textbf{EasyAR} & \checkmark & \checkmark & \checkmark & \checkmark & – & \checkmark & Grabación de pantalla \\
\hline
\textbf{AR Foundation} & \checkmark & \checkmark & \checkmark & \checkmark & \checkmark & \checkmark & \\
\hline
    \end{tabular}
  }
    \caption{Comparación de funcionalidades}
    \label{tab:funcionalidades}
\end{table}
Como vemos en la tabla, menos en el reconocimiento de rostro, casi todas las librerías tienen las mismas funcionalidades, por lo que a priori nos sirve cualquiera para realizar nuestras pruebas de concepto, pero antes toca probarlas y ver cuál es la librería que mejor implementadas y pulidas tiene estas funcionalidades.

\subsection{Usabilidad}
Este apartado consistirá en dos apartados, en la primera compararemos en que plataformas se pueden ejecutar las librerías, y luego comparemos en que plataforma y lenguajes se pueden programar las aplicaciones.

\begin{table}[ht]
\resizebox{\textwidth}{!} {
    \centering
    \begin{tabular}{c|c|c|c|c|c|c|c}
       SDK &	Unity3D (Android, iOS) &	Unreal Engine 4 &	Java &	Objective-C &	C++ & JavaScript \\
       \hline
Wikitude & \checkmark & – & \checkmark & \checkmark & – & \checkmark \\
\hline
ARKit & \checkmark & \checkmark & – & \checkmark & – & – \\
\hline
ARCore & \checkmark & \checkmark & \checkmark & \checkmark & – & – \\
\hline
Vuforia & \checkmark & – & \checkmark & \checkmark & \checkmark & – \\
\hline
Kudan & \checkmark & – & \checkmark & \checkmark & – & – \\
\hline
MaxST & \checkmark & – & \checkmark & – & \checkmark & – \\
\hline
8th Wall  & \checkmark & – & – & – & – & \checkmark \\
\hline
EasyAR & \checkmark & – & \checkmark & \checkmark & \checkmark & – \\
\hline
AR Foundation & \checkmark & – & – & – & – & – \\
\hline
    \end{tabular}
  }
    \caption{Comparación de plataformas y lenguajes soportados}
    \label{tab:plataformas}
\end{table}

Podemos ver que prácticamente todas las librerías soportan Android e iOS, con lo que podemos deducir que es donde más se está invirtiendo en el mercado. Quizás las Smart Glasses puedan brindar una experiencia de realidad aumentada más agradable, pero todavía está muy lejos de ser accesible para la mayoría de la población, mientras que un dispositivo móvil es mucho más asequible.

Con esta tabla, es muy fácil ver que Unity3D sale ganador, podemos trabajar con absolutamente todas las librerías, además, gracias a su editor y entorno visual, resulta muchísimo más cómodo y ahorra mucho tiempo a la hora de hacer una aplicación de realidad aumentada.

\section{Conclusiones prueba de concepto}
\begin{table}[H]
\resizebox{\textwidth}{!} {
    \centering
    \begin{tabular}{m{3cm}|c|c|c|c|c|c|c|c|c}
    &Wikitude&	ARKit &	ARcore & Vuforia &	MaxST &	EasyAR & Kudan & 8th Wall XR & ARFoundation\\
     \hline
         Calidad de la documentación      &          &       &        &         &       &        &       &             &              \\
 \hline
 Distancia máxima de captura (m)  &          &       &        &         &       &        &       &             &              \\
  \hline
Estabilidad del punto de anclaje &          &       &        &         &       &        &       &             &              \\
 \hline
Comportamiento con luz intensa   &          &       &        &         &       &        &       &             &              \\
 \hline
Comportamiento con luz ambiente  &          &       &        &         &       &        &       &             &              \\
 \hline
Comportamiento con luz tenue     &          &       &        &         &       &        &       &             &              \\
 \hline
Estimación de luces              &          &       &        &         &       &        &       &             &              \\
 \hline
Total (puntuación)               &          &       &        &         &       &        &       &             &            
    \end{tabular}
}
    \caption{Análisis de las características de las librerías de RA sin marcadores}
    \label{tab:my_label}
\end{table}

\noindent