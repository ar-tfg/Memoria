\chapter{Conclusiones y trabajo futuro}
Gracias a este estudio podemos hacer una estimación bastante certera de cuál es el estado, las capacidades y las limitaciones de las librerías y la tecnología de realidad aumentada en la actualidad cumpliendo de esta manera nuestro objetivo principal con creces.\\

La realidad aumentada sin marcadores es una tecnología muy nueva, además es accesible a un gran número de usuario gracias a los \textit{smartphones}. En los últimos años se han desarrollado diferentes librerías que facilitan el desarrollo de las aplicaciones con RA sin marcadores. Dichas librerías se actualizan frecuentemente introduciendo nuevas características y mejorando las existentes con el fin de enriquecer la experiencia. Tras la finalización del proyecto, podemos extraer varias conclusiones sobre las diferentes librerías expuestas en este trabajo, proporcionadas tanto por los resultados obtenidos como por la investigación llevada a cabo durante estos meses.\\

Gracias a las pruebas desarrolladas con las principales librerías del mercado podemos establecer que Google y Apple lideran en calidad, por lo que ARCore y ARKit, son una de las opciones más viables para una aplicación que exige estabilidad y precisión. También cabe destacar el esfuerzo por parte del equipo de desarrollo de Unity3D para implementar la API de ARFoundation, que nos permite desarrollar aplicaciones de realidad aumentada para Android e iOS con el mismo código. \\

Por otra parte, tenemos librerías que son más inestables, pero son capaces de funcionar en casi todos los dispositivos~\cite{wikitudeInstant} y en un gran porcentaje de entornos y condiciones. Esto hace que el uso de estas librerías, como pueden ser Vuforia, EasyAR, o Maxst, sean una opción a elegir para aplicaciones que no requieran una estabilidad tan minuciosa.\\

Después de haber probado todas las librerías vistas anteriormente, queda de manifiesto que se ha conseguido que la estabilidad sea prácticamente perfecta en la mayoría de los casos. También observamos un gran avance en la estimación de luces, muy lograda en casos como ARCore y ARkit, aportando mayor realismo a la experiencia inmersiva. Concluimos que los desarrolladores de ambas empresas líderes (Apple y Google) están optando por revolucionar el mercado con tecnologías como la oclusión y los \textit{cloud anchors}. La realidad aumentada es una tecnología muy novedosa, por lo que avanza muy rápidamente, aproximadamente cada mes y medio ARCore recibe una nueva actualización. De hecho, durante el desarrollo de este trabajo, las librerías sufrieron actualizaciones importantes con la inclusión de nuevas características.\\

Gracias a la extensa documentación existente y a los tutoriales creados por la comunidad, el aprendizaje del desarrollo de aplicaciones con realidad aumentada sin marcadores resulta cómodo y accesible para el desarrollador. Esta accesibilidad nos ha permitido el desarrollo de tres aplicaciones destinadas a tres sectores diferentes y nos ha resuelto problemas y dudas que han surgido durante el desarrollo de las mismas.


\section{Futuros pasos}

Como futuros pasos se puede extender el \textbf{análisis a las funcionalidades de las \textit{smartglasses}} de realidad aumentada. Estos dispositivos dominan actualmente el mercado y es interesante someterlos a las mismas pruebas y sacar las posibilidades y limitaciones de cada uno. \\

Por otro lado, todas las pruebas que hemos realizado nosotros han sido en interiores, por lo que se deberían de realizar pruebas en un espacio abierto en el exterior y comprobar el comportamiento.\\

Además es importante analizar la experiencia de usuario en la realidad aumentada. Es unos de los problemas que hemos identificado, ya que la mayoría de los usuarios que prueban por primera vez una aplicación de realidad aumentada suelen estar desorientados impidiéndoles disfrutar de la experiencia.\\

Para complementar nuestro trabajo, es interesante hacer un estudio sobre otras funcionalidades que soportan algunas librerías, como el reconocimiento facial, la oclusión, los cloud anchors. Con este tipo de investigación se podría ampliar nuestro trabajo integrando más funcionalidades en el documento de evaluación de las librerías de realidad aumentada.

\noindent 
