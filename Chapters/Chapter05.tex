\chapter*{Conclusiones}
\addcontentsline{toc}{chapter}{Conclusiones}

Tras la finalización del proyecto, podemos extraer varias conclusiones sobre las diferentes librerías expuestas en este trabajo, proporcionados tanto por los resultados obtenidos como por la investigación llevada a cabo durante estos meses.\\

Gracias a este estudio podemos hacer una estimación bastante certera de cuál es el estado, las capacidades y las limitaciones de las librerías y la tecnología de realidad aumentada actualmente cumpliendo de esta manera nuestro objetivo principal con creces.\\

El proyecto ha aportado nuevas conclusiones respecto al estado de la cuestión y los antecedentes de los que se parte.\\

Gracias a las pruebas desarrolladas con las principales librerías del mercado podemos establecer que Google y Apple lideran en calidad, por lo que ARCore y ARKit, son una de las opciones más viables para una aplicación que exige estabilidad y precisión. También cabe destacar el esfuerzo por parte del equipo de desarrollo de Unity3D para implementar la API de ARFoundation, que nos permite desarrollar aplicaciones de realidad aumentada para Android e iOS con el mismo código. 
Por otra parte, tenemos librerías ``low cost'' que son más inestables, pero son capaces de funcionar en casi todos los dispositivos \cite{wikitudeInstant} y en un gran porcentaje de entornos y condiciones. Esto hace que el uso de estas librerías, como puede ser Vuforia, EasyAR, o Maxst, sea una opción a elegir para aplicaciones que no requieran una estabilidad tan minuciosa.\\

Después de haber probado todas las librerías vistas anteriormente, queda de manifiesto que se ha conseguido que la estabilidad sea prácticamente perfecta en la mayoría de los casos. También observamos un gran avance en la estimación de luces, muy lograda en casos como ARCore y ARkit, aportando mayor realismo a la experiencia inmersiva. Concluimos que los desarrolladores de ambas empresas líderes (Apple y Google) están optando por revolucionar el mercado con tecnologías como la oclusión y los \textit{Cloud Anchors}. La realidad aumentada es una tecnología muy novedosa, por lo que avanza muy rápidamente, aproximadamente cada mes y medio ARCore recibe una nueva actualización, sin ir más lejos al realizar este trabajo hemos podido observar muchos cambios importantes en el funcionamiento de la librería.\\

Gracias a la extensa documentación existente y a los tutoriales creados por la comunidad, el aprendizaje del desarrollo de aplicaciones con realidad aumentada sin marcadores resulta cómodo y accesible para el desarrollador. Esta accesibilidad nos ha permitido el desarrollo de tres aplicaciones destinadas a tres sectores diferentes.


\section*{Futuros pasos}
\addcontentsline{toc}{section}{Futuros pasos}

Como futuros pasos podemos establecer seguir pendientes de la evolución de las librerías de cara a poder generar aplicaciones de mayor calidad. Por otro lado nos gustaría retomar algunas ideas que descartamos y seguir con el desarrollo de las pruebas de concepto, ya que nos parecen de gran utilidad y el videojuego creado utiliza mecánicas nuevas nunca vistas. Las pruebas que hemos realizado siempre han sido en interior, nos gustaría probar el funcionamiento de las librerías en exterior y ver cómo se comportan.

\noindent 
