\parindent=0em
\chapter*{Introducción}
\addcontentsline{toc}{chapter}{Introducción}
\noindent
\section*{Motivación}
\addcontentsline{toc}{section}{Motivación}
En los últimos años la realidad aumentada se ha convertido en una tecnología madura y presente en la mayoría de los usuarios. Apoyada por el auge de dispositivos móviles inteligentes y la mejora de los componentes de estos la realidad aumentada es adoptada cada vez más por un mayor público. Se trata de un tema de actualidad recurrente y de referencia en múltiples productos de innovación.\vspace{\baselineskip}
Se estima que el tamaño del mercado de la RA crecerá de 3.5 mil millones en el 2017 a más de 198 mil millones de dólares en el 2025.\cite{Statista} En los años próximos se espera que revolucione mercados como son el arte, la educación, la publicidad, procesos de fabricación y montaje, turismo y especialmente el mundo de los videojuegos entre otros. Debido a este gran crecimiento del sector es un excelente tema para tratar de cara a conocer las limitaciones y puntos a destacar de cada una de las tecnologías existentes en la actualidad.

\section*{Objetivos}
\addcontentsline{toc}{section}{Objetivos}
El objetivo principal del desarrollo de este proyecto será conocer las posibilidades y limitaciones de las librerías de realidad aumentada sin marcadores existentes en la actualidad. En base a este objetivo se fijaron los siguientes objetivos específicos:
\begin{enumerate}
\item Investigación de las principales librerías de realidad aumentada sin marcadores.
\item Implementación de pruebas de concepto de carácter básico de cada una de ellas para analizar los pros y contras.
\item Análisis de los resultados obtenidos de las pruebas de concepto.
\item Planteamiento e implementación de diferentes aplicaciones de realidad aumentada sin marcadores en función de los resultados obtenidos en el análisis.
\item Desarrollar una aplicación/concepto de realidad aumentada en multijugador con los cloud anchors.
\end{enumerate}

\section*{Metodología}
\addcontentsline{toc}{section}{Metodología}
Para llevar a cabo estos objetivos se investigará a través de fuentes en internet, artículos científicos, estudios previos y libros, todos ellos reflejados en la bibliografía y webgrafía. Estos recursos serán la base de la fundamentación del proyecto y por ello se abrirán dos vías de investigación principales: se analizará por una parte acerca de las diferentes librerías de desarrollo en realidad aumentada y por otra las diferentes aplicaciones en el mercado de esta tecnología.\\

La revisión bibliográfica que se llevará a cabo vendrá definida por las dos áreas del conocimiento que se deben investigar de cara a desarrollar el objetivo principal planteado. En el campo del conocimiento técnico de la realidad aumentada se investigará a través de bibliografía recomendada por profesores de la Facultad de Informática de la Complutense. Los libros más significativos por tratar serán “\textit{Handbook of Augmented Reality}” de Borko Furht, “\textit{Augmented reality games I, Understanding the Pokémon GO phenomenon}” de Vladimir Geroimenko y “\textit{Augmented reality games II, The gamification of education, medicine and art}” de Vladimir Geroimenko.\\

Debido al continuo cambio que experimentan las tecnologías de realidad aumentada gran parte de la investigación se verá supeditada a artículos científicos, así como a la documentación de las distintas librerías que lideran el mercado.\\

Una vez completada la investigación teórica se testearán diferentes aplicaciones ya existentes con el objetivo de encontrar sus fortalezas y debilidades. A través de estas conclusiones se podrá llevar a cabo una prueba de concepto más solida y veraz evitando cometer errores anteriormente observados.\vspace{\baselineskip}
Se buscará conocer las características específicas de las librerías escogidas delimitando los pros y contras de cada una de ellas. Para identificarlas se realizará un test definido y cerrado, poniendo a prueba las diferentes librerías en el mismo dispositivo de cara a establecer una comparativa entre todas. Estas aplicaciones test se desarrollarán cuando sea posible en el entorno Unity.\\

Por último, se desarrollarán tres aplicaciones de mayor nivel de complejidad que nos permitan explotar las virtudes de tres librerías diferentes de cara a mostrar y fijar las conclusiones extraídas del anterior estudio.\\

A continuación, se detallan las tecnologías que se utilizarán a lo largo del desarrollo del proyecto.\\

Para la creación de las aplicaciones, se utilizará como entorno de desarrollo principalmente \textbf{Unity 2019.2}, uno de los motores de videojuegos punteros y referentes en la industria. Gracias a su versatilidad e interfaz intuitiva nos permitirá iterar rápidamente a lo largo de los test y pruebas de concepto. Este motor es uno de los escogidos por la facultad para estudiar a lo largo del grado de desarrollo de videojuegos con lo que nos resultará más cómodo y familiar, favoreciendo de nuevo la agilidad en el desarrollo.\vspace{\baselineskip}

Como IDE\footnote{IDE: Integrated Development Environment (Entorno de desarrollo integrado)}  se utilizará \textbf{Visual Studio 2019} acompañado por Visual Studio Tools para Unity una extensión gratuita de Visual Studio que lo convierte en una completa herramienta con la que desarrollar aplicaciones y juegos multiplataforma con Unity. Esta herramienta permite la integración de Visual Studio con el editor de Unity haciendo más eficaz el desarrollo.\\

El sistema de control de versiones escogido será \textbf{Github} ya que estamos habituados a la plataforma y nos permite integrarlo con herramientas como Visual Studio y Microsoft Teams.\\

La herramienta de seguimiento de tareas y comunicación entre el equipo escogida será \textbf{Microsoft Teams} dada su versatilidad y posibilidad de añadir herramientas.\\

Gracias a todas estas metodologías tanto prácticas, teóricas y técnicas y a las se lograrán cumplir los objetivos propuestos.\\

\section*{Plan de trabajo}
\addcontentsline{toc}{section}{Plan de trabajo}
La primera parte del trabajo consistirá en informarnos e investigar sobre las librerías de RA sin marcadores que lideran el mercado esta parte se realizará conjuntamente por los tres. Se estudiarán las diferentes tecnologías que componen la experiencia de la RA en los dispositivos móviles para entender cómo funciona a bajo nivel y estar actualizados con la demanda del mercado de cara a poder hacer una prueba de concepto verosímil con las aplicaciones de RA actuales. \vspace{\baselineskip}

En una segunda parte, después de identificar las librerías que existen en el mercado, se probarán (las que sean posibles) teniendo en cuenta los dispositivos y plataformas soportadas por cada una de ellas, a su vez se tendrá en cuenta la existencia de una licencia gratuita o de prueba. Se desarrollarán diversas aplicaciones de carácter básico a modo de test, éstas nos permitirán poner a prueba cada una de las principales librerías de realidad aumentada sin marcadores, comprobando su eficiencia. Los test serán ejecutados en las mismas condiciones lumínicas y con el mismo dispositivo de cara a obtener una mayor precisión en la comparación. En este caso nos dividiremos el desarrollo de los test de manera equitativa.\vspace{\baselineskip}

Una vez que se hayan encontrado las librerías que mejor se ajusten a nuestras necesidades, se desarrollarán las siguientes pruebas de concepto:
\begin{itemize}
\item Montar muebles de Ikea: se mostrará el proceso de montaje de un mueble en realidad aumentada. Ayudando al usuario a montar cada una de las piezas del proceso.
\item Juego multijugador usando \textit{cloud anchors}: se desarrollará un juego en el que poder poner a prueba las tecnologías de realidad aumentada sin marcadores combinada con puntos de localización en la nube permitiendo crear un juego multijugador.
\item Visualizador de modelos 3D con las gafas Aryzon: se hará uso de unas gafas de realidad aumentada tipo \textit{cardboard} en las que podremos ver superpuesto un modelo 3D, se podrá interactuar con el modelo 3D a través del mando de Xbox permitiendo rotarlo, escalarlo y moverlo.
\end{itemize}