\chapter{Pruebas de concepto}
\section{Propuestas}
Una vez se han testeado y analizado las diferentes tecnologías surgen diferentes ideas como prueba de concepto en función de las características particulares de cada tecnología. Las cuales son enumeradas a continuación:

\begin{itemize}
\item Desarrollo de un \textit{plugin} capaz de gestionar los \textit{cloud anchor} en interiores y almacenar la información durante un período de tiempo ajustable a las necesidades del producto. Será capaz de identificar estas posiciones \textit{online} y \textit{offline} mediante una base de datos almacenada tanto en servidor como en local si fuese necesario. Plataformas deseables: Android (prioritario) y iOS. 
\item \textit{High level }API que permite desarrollar de manera más versátil y sencilla aplicaciones en realidad aumentada teniendo compatibilidad plena entre ARKit y ARCore. El usuario será capaz de desarrollar una aplicación en realidad aumentada completa sin apenas líneas de código ya sea mediante \textit{blueprints} o módulos.
\item Juego multijugador con \textit{Cloud Anchors}. Juego en el que dos jugadores o más compiten por conseguir más puntos por destruir edificios. (Inspirado en Bombardero - Amstrad CPC)
\item Plataforma de realidad aumentada: permitirá al usuario acceder en el momento a diferentes contenidos ya sean vídeos, juegos o experiencias sin necesidad de salir de la aplicación. Esta idea estaba pensada para el uso con unas gafas de AR y mando inalámbrico para el control del dispositivo. Inspirado en Google DayDream.
\item Reconocimiento de objetos.
\item Proceso de montaje de muebles paso a paso del en realidad aumentada.
\end{itemize}

\section{Juego multijugador con cloud anchor (ARCore)}
\section{Instrucciones de montaje de muebles en AR}
\section{Visualizador de objetos en AR con gafas Aryzon}


\noindent
