\parindent=0em
\addcontentsline{toc}{chapter}{Introduction}
\chapter*{Introduction}
\noindent
\section*{Motivation}
\addcontentsline{toc}{section}{Motivation}
Over the last few years, the augmented reality has become a mature technology and its present in most users. Supported by the rise of smartphones and the improvement of their components, augmented reality is increasingly being adopted by a larger audience. That’s why this is an important and recurring theme in innovation products.

The augmented reality market is expected to grow from 3.5 million in 2017 to more than 198 million dollars by 2025. In the next few years it is expected to transform several markets like art, education, advertising, manufacturing processes, tourism and specially videogames among others. It’s an outstanding theme due to the growth of this sector and that’s why we decided to study the limitations and capabilities of the currently available technologies.


\section*{Objetives}
\addcontentsline{toc}{section}{Objetives}
The main goal of the thesis will be determining and studying the capabilities and limitations of the currently available markerless augmented reality libraries. In order to accomplish this, we stablished the following objectives:

\begin{itemize}
\item Discover and study cutting-edge markerless augmented reality libraries.
\item Deployment of simple proof of concept with each library in order to find its pros and cons.
\item Analysis of the results achieved by the previous proofs of concept.
\item Approach and implementation of different markerless augmented reality applications based on the results previously achieved.
\item Develop a multiplayer augmented reality application or proof of concept using Cloud Anchors.
\end{itemize}


\newpage
\section*{Work methodology}
\addcontentsline{toc}{section}{Work methodology}
To carry out these goals, we will investigate through internet sources, scientific articles, previous studies and books; all of them are reflected in the bibliography and webgraphy. These resources will be the basis of the project's foundation and that is why two main research channels will be opened: on one hand we will research and analyze the different augmented reality development libraries and on the other the different applications in the market for this technology.\\


The bibliographical review that will be carried out will be defined by two areas of knowledge that must be investigated in order to develop the main objective. In order to acquire technical knowledge of augmented reality, we will investigate through bibliography recommended by professors of the Complutense. The most relevant books will be Borko Furht's “Handbook of Augmented Reality” and Vladimir Geroimenko‘s “Augmented reality games I, Understanding the Pokémon GO phenomenon” and “Augmented reality games II, The gamification of education, medicine and art”.\\

Due to the continuous change experienced by augmented reality technologies, the research will be strongly bounded to scientific articles, as well as to the documentation of the different libraries that lead the market.\\

Once the bibliographical research is completed, different existing applications will be tested in order to find their strengths and weaknesses. Thanks to these conclusions we will carried out a more solid and truthful proof of concept, avoiding mistakes previously observed.\\

We will seek to know the specific characteristics of the chosen libraries delimiting the pros and cons of each of them. To identify them, we will perform several defined and closed tests, testing the different libraries in the same device in order to establish a comparison between them. These test applications will be developed whenever possible in the Unity environment.\\

Finally, three applications of greater complexity will be developed that will allow us to exploit the virtues of different libraries in order to show and stablish the previous study conclusions.\\

The technologies that will be used throughout the project development are detailed below.\\


Unity 2019.2 and 2018.3 will be used as our development environment, as it is one of the leading video game engines and it’s a reference in the industry. Thanks to its versatility and intuitive interface it will allow us to iterate quickly throughout the tests and proofs of concept. We will be more comfortable with this engine as it is the one chosen by the faculty to study throughout the degree of videogame development, facilitating again the agility in development.\\

Visual Studio 2019 will be used as our IDE, accompanied by Visual Studio Tools for Unity, a free Visual Studio extension that makes it a complete tool with which to develop applications and multiplatform games with Unity. This tool allows the Visual Studio integration with Unity editor allowing us to develop more efficiently.\\

Github will be the system control version chosen, we are used to the platform and allows us to integrate it with tools such as Visual Studio and Microsoft Teams.\\

The task tracking and communication tool between the team will be Microsoft Teams given its versatility and the possibility of adding tools.
Thanks to all these methodologies both practical, theoretical and technical we will achieve the objectives set.\\


\section*{Work plan}
\addcontentsline{toc}{section}{Work plan}
First of all we will work on researching and analysing cutting-edge markerless AR libraries. This part will be carried out jointly by the three of us. We will study the different technologies that make up the experience of AR on smartphones to understand how it works at a low level and so we can figure out the market demand in order to be able to make a credible proof of concept with current RA applications.\vspace{\baselineskip}

Secondly, after identifying the libraries that exist in the market, those will be tested (those that are possible) considering the devices and platforms supported by each of them, also keeping in mind the licenses offered in each of them.

We will develop as tests basic applications, these will allow us to evaluate each of the markerless augmented reality libraries, checking their efficiency. The tests will be executed in the same light conditions and with the same device in order to obtain a greater precision in the comparison. In this case we will divide the test development in an equitable manner.\vspace{\baselineskip}

Once the libraries that best fit our needs have been found, the following proof of concept will be developed:
\begin{itemize}
\item \textbf{Assemble Ikea furniture}: the process of assembling a piece of furniture in augmented reality will be displayed. Helping the user in the assembling process.
\item \textbf{Multiplayer game}: we will develop a game in which we will combine markerless technology with cloud anchors allowing us to create a multiplayer game.
\item \textbf{3D models visualizer with Aryzon glasses}: we will use cardboard augmented reality glasses in which we can see a 3D model overlapped with the real world, in the application the user will interact with the 3D model through the Xbox controller allowing to rotate, scale and move it.
\end{itemize}
























