\newpage
\chapter*{Resumen}
\addcontentsline{toc}{chapter}{Resumen}
La realidad aumentada (RA) se ha convertido en la última década en una tecnología accesible a millones de usuarios a través de sus dispositivos móviles. Basada inicialmente en el uso de marcadores, hoy en día existen algoritmos que permiten su uso sin marcadores que, pese al éxito de juegos como Pokemon Go, no han terminado de despegar de manera general.\\

Este trabajo de fin de grado plantea un análisis de las posibilidades de la realidad aumentada sin marcadores. A lo largo del proyecto se exploran y estudian las librerías existentes, de cara a descubrir sus similitudes y diferencias realizando pequeñas pruebas de concepto. Con los resultados obtenidos de este estudio se han realizado tres aplicaciones que explotan las características de las librerías y las tecnologías estudiadas. La primera aplicación que se desarrolló ayuda al montaje de muebles en realidad aumentada. En esta aplicación se puso a prueba la librería ARCore y sus funcionalidades sin marcadores. En segundo lugar, se ha desarrollado un videojuego multijugador de cara a mostrar sus posibilidades en el campo haciendo uso de los puntos de ancla en la nube y la detección de planos. Por último, se ha creado una aplicación que permite disfrutar de la experiencia de una manera más natural usando las gafas de Aryzon\footnote{Gafas \textit{cardboard} que permiten tener una experiencia de realidad aumentada~\cite{Aryzon}.} interactuando con los objetos colocados en el plano escogido.\\
\\
\\

\section*{Palabras clave}
\addcontentsline{toc}{section}{Palabras clave}
Realidad aumentada, SLAM, ARCore, ARKit, RA sin marcadores, multijugador, seguimiento, detección de planos.

\noindent