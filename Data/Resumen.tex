\newpage
\chapter*{Resumen}
\addcontentsline{toc}{chapter}{Resumen}
La realidad aumentada se ha convertido en la última década en una tecnología accesible a millones de usuarios a través de sus dispositivos móviles. Basada inicialmente en el uso de marcadores, hoy en día existen algoritmos que permiten su uso sin marcadores que, pese al éxito de juegos como Pokemon Go, no han terminado de despegar de manera general.\\

Este Trabajo de Fin de Grado plantea un análisis de las posibilidades de la realidad aumentada, concretamente sin marcadores. A lo largo del proyecto se exploran y estudian las librerías existentes, de cara a descubrir sus similitudes y diferencias realizando pequeñas pruebas de concepto. Con los resultados obtenidos de este estudio se plantearán tres aplicaciones que exploten las características de las librerías y las tecnologías estudiadas. Una primera aplicación de montaje de muebles en realidad aumentada donde se pone a prueba la librería ARCore y sus funcionalidades sin marcadores. De cara a mostrar sus posibilidades en el campo de los videojuegos se desarrolla un videojuego multijugador haciendo uso de los puntos de ancla en la nube y la detección de planos. Por último, se crea una aplicación que nos permita interactuar con el medio de una manera más natural con unas gafas Aryzon\footnote{Gafas \textit{cardboard} que permiten tener una experiencia de realidad aumentada.\cite{Aryzon}} interactuando con los objetos colocados en el plano escogido.\\
\\
\\

\section*{Palabras clave}
\addcontentsline{toc}{section}{Palabras clave}
Realidad aumentada, SLAM, ARCore, ARKit, RA sin marcadores, multijugador, tracking, detección de planos, 

\noindent