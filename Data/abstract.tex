\newpage
\chapter*{Summary}
\addcontentsline{toc}{chapter}{Abstract}
In the last decade augmented reality (AR) has become an accessible technology to millions of users through their smartphones. Initially based on the use of markers, today there are algorithms capable of using it without markers. Despite the success of games like Pokemon Go, this technology has not yet been consolidated.\\

This thesis presents an analysis of the possibilities of augmented reality, focused on markerless AR. Throughout this project, existing libraries are explored and studied, in order to discover their similarities and differences by developing small proofs of concept.\\ 

With the results obtained from this study, three applications have been developed. These applications try to leverage all the potential of the libraries and the technologies studied. The first application helps the user in the furniture assembling process, showing it step by step, using ARCore and its markerless functionalities. Secondly, in order to show the AR capabilities in videogames, a multiplayer videogame was developed using cloud anchors and plane detection.
Finally, the last application allows us to visualize the experience in a more natural way using Aryzon\footnote{Cardboard glasses that allow us to have an AR experience~\cite{Aryzon}.} glasses interacting with the objects placed in the chosen plane.
\\
\\
\\

\section*{Keywords}
\addcontentsline{toc}{section}{Keywords}
Augmented reality, SLAM, ARCore, ARKit, markerless AR, multiplayer, tracking, plane detection.
\noindent
\noindent

